\documentclass[12pt, a4paper]{article}

% -------------------------------------------------------
% Packages
% -------------------------------------------------------
\usepackage{geometry}
\geometry{margin=1in}

\usepackage{fancyhdr}
\usepackage{titlesec}
\usepackage{listings}
\usepackage{xcolor}
\usepackage{graphicx}

% -------------------------------------------------------
% Header & Footer
% -------------------------------------------------------
\pagestyle{fancy}
\fancyhf{}
\lhead{Kamithkar Vinod}
\rhead{DBT - Assignment 8}
\cfoot{\thepage}

% -------------------------------------------------------
% Section Formatting
% -------------------------------------------------------
\titleformat{\section}
  {\large\bfseries}
  {Problem \thesection:}
  {0.5em}{}

\titleformat{\subsection}[runin]
  {\bfseries}
  {Code:}
  {0.5em}{}[---]

\titleformat{\subsubsection}[runin]
  {\bfseries}
  {Output:}
  {0.5em}{}[---]

% -------------------------------------------------------
% SQL Syntax Highlighting
% -------------------------------------------------------
\lstdefinelanguage{SQL}{
  morekeywords={
    SELECT, FROM, WHERE, GROUP, BY, ORDER, ASC, DESC, JOIN, ON, AS,
    AND, OR, NOT, IN, IS, NULL, LIKE, HAVING, COUNT, SUM, AVG, MIN, MAX,
    CREATE, TABLE, INSERT, INTO, VALUES, UPDATE, SET, DELETE, DISTINCT,
    CASE, WHEN, THEN, ELSE, END, BETWEEN, EXISTS, UNION, ALL, ANY, LEFT,
    RIGHT, INNER, OUTER, LIMIT, OFFSET, PROCEDURE, BEGIN, FUNCTION, END,
    RETURNS, RETURN, TRIGGER, BEFORE, AFTER, FOR, EACH, ROW, SIGNAL
  },
  sensitive=false,
  morecomment=[l]{--},
  morestring=[b]',
}

\lstset{
  language=SQL,
  basicstyle=\ttfamily\small,
  keywordstyle=\color{blue}\bfseries,
  commentstyle=\color{gray}\itshape,
  stringstyle=\color{red},
  showstringspaces=false,
  frame=single,
  breaklines=true,
  numbers=none
}

% -------------------------------------------------------
% Document Start
% -------------------------------------------------------
\begin{document}

% -------------------------------------------------------
% Title Page
% -------------------------------------------------------
\begin{center}
    \LARGE \textbf{Assignment - 8} \\[0.5cm]
    \Large \textbf{Database Management Systems (DBMS)} \\[1cm]

    \begin{tabular}{rl}
        \textbf{Name:} & Kamithkar Vinod \\
        \textbf{Course:} & PG DAC August 2025 \\
        \textbf{PRN:} & 250850320040 \\
        \textbf{Form No:} & 250500480 \\
        \textbf{Date:} & 30-10-2025 \\
    \end{tabular}
\end{center}

\vspace{1cm}
\hrule
\vspace{0.5cm}

% -------------------------------------------------------
% Problems
% -------------------------------------------------------

% 1
\section{Before\_Insert}
\textbf{Task:} Create a trigger to automatically store employee names in uppercase before inserting into the table.

\subsection{}
\begin{lstlisting}
DELIMITER $$
CREATE TRIGGER before_emp_insert_upper
BEFORE INSERT ON employees
FOR EACH ROW
BEGIN
    SET NEW.emp_name = UPPER(NEW.emp_name);
END$$
DELIMITER ;

INSERT INTO employees(emp_name, salary) VALUES ('john doe', 40000);
SELECT * FROM employees;
\end{lstlisting}

\subsubsection{}
\begin{center}
    \includegraphics[width=0.8\textwidth]{1.png}
\end{center}

% 2
\section{Before\_Insert}
\textbf{Task:} Create a trigger that ensures no employee salary is less than 10000 when inserting data.

\subsection{}
\begin{lstlisting}
DELIMITER $$
CREATE TRIGGER before_emp_insert_min_salary
BEFORE INSERT ON employees
FOR EACH ROW
BEGIN
 IF NEW.salary < 10000 THEN
   SET NEW.salary = 10000;
 END IF;
END$$
DELIMITER ;

-- Example to try
INSERT INTO employees(emp_name, salary) VALUES ('mary', 5000);
SELECT * FROM employees;
\end{lstlisting}

\subsubsection{}
\begin{center}
    \includegraphics[width=0.8\textwidth]{2.png}
\end{center}

% 3
\section{After\_Insert}
\textbf{Task:} Create a trigger that records every new employee addition in the audit\_log table.

\subsection{}
\begin{lstlisting}
DELIMITER $$
CREATE TRIGGER after_emp_insert_log
AFTER INSERT ON employees
FOR EACH ROW
BEGIN
 INSERT INTO audit_log(action, emp_id, new_salary)
 VALUES ('INSERT', NEW.emp_id, NEW.salary);
END$$
DELIMITER ;

-- Example to try
INSERT INTO employees(emp_name, salary) VALUES ('david', 35000);
SELECT * FROM audit_log;

\end{lstlisting}

\subsubsection{}
\begin{center}
    \includegraphics[width=0.8\textwidth]{3.png}
\end{center}

% 4
\section{After\_Insert}
\textbf{Task:} Create a trigger that displays a welcome message when a new employee is inserted.

\subsection{}
\begin{lstlisting}
-- Write your SQL code here
\end{lstlisting}

\subsubsection{}
\begin{center}
    % \includegraphics[width=0.8\textwidth]{4.png}
\end{center}

% 5
\section{Before\_Update}
\textbf{Task:} Create a trigger that prevents salary reduction for any employee.

\subsection{}
\begin{lstlisting}
DELIMITER $$
CREATE TRIGGER before_emp_update_no_decrease
BEFORE UPDATE ON employees
FOR EACH ROW
BEGIN
 IF NEW.salary < OLD.salary THEN
   SET NEW.salary = OLD.salary;
 END IF;
END$$
DELIMITER ;

-- Example to try
UPDATE employees SET salary = 10000 WHERE emp_id = 1;
SELECT * FROM employees;
\end{lstlisting}

\subsubsection{}
\begin{center}
    \includegraphics[width=0.8\textwidth]{5.png}
\end{center}

% 6
\section{Before\_Update}
\textbf{Task:} Create a trigger that rounds off the updated salary to the nearest integer.

\subsection{}
\begin{lstlisting}
DELIMITER $$
CREATE TRIGGER before_emp_update_round
BEFORE UPDATE ON employees
FOR EACH ROW
BEGIN
 SET NEW.salary = ROUND(NEW.salary, 0);
END$$
DELIMITER ;

-- Example to try
UPDATE employees SET salary = 45789.65 WHERE emp_id = 2;
SELECT * FROM employees;
\end{lstlisting}

\subsubsection{}
\begin{center}
    \includegraphics[width=0.8\textwidth]{6.png}
\end{center}

% 7
\section{After\_Update}
\textbf{Task:} Create a trigger that logs old and new salaries whenever an employee’s salary is updated.

\subsection{}
\begin{lstlisting}
DELIMITER $$
CREATE TRIGGER after_emp_update_log
AFTER UPDATE ON employees
FOR EACH ROW
BEGIN
 INSERT INTO audit_log(action, emp_id, old_salary, new_salary)
 VALUES ('UPDATE', NEW.emp_id, OLD.salary, NEW.salary);
END$$
DELIMITER ;

-- Example to try
UPDATE employees SET salary = 48000 WHERE emp_id = 1;
SELECT * FROM audit_log;
\end{lstlisting}

\subsubsection{}
\begin{center}
    \includegraphics[width=0.8\textwidth]{7.png}
\end{center}

% 8
\section{After\_Update}
\textbf{Task:} Create a trigger that displays a message whenever an employee’s salary is increased.

\subsection{}
\begin{lstlisting}
-- Write your SQL code here
\end{lstlisting}

\subsubsection{}
\begin{center}
    % \includegraphics[width=0.8\textwidth]{8.png}
\end{center}

% 9
\section{Before\_Delete}
\textbf{Task:} Create a trigger that prevents deletion of a record if the employee name is “CEO”.

\subsection{}
\begin{lstlisting}
DELIMITER $$
CREATE TRIGGER before_emp_delete_ceo
BEFORE DELETE ON employees
FOR EACH ROW
BEGIN
 IF OLD.emp_name = 'CEO' THEN
   SIGNAL SQLSTATE '45000'
   SET MESSAGE_TEXT = 'Cannot delete CEO record!';
 END IF;
END$$
DELIMITER ;

-- Example to try
INSERT INTO employees(emp_name, salary) VALUES ('CEO', 90000);
DELETE FROM employees WHERE emp_name = 'CEO'; -- should raise error
\end{lstlisting}

\subsubsection{}
\begin{center}
    \includegraphics[width=0.8\textwidth]{9.png}
\end{center}

% 10
\section{Before\_Delete}
\textbf{Task:} Create a trigger that logs any attempt to delete an employee record in the audit\_log table.

\subsection{}
\begin{lstlisting}
DELIMITER $$
CREATE TRIGGER before_emp_delete_log
BEFORE DELETE ON employees
FOR EACH ROW
BEGIN
 INSERT INTO audit_log(action, emp_id, old_salary)
 VALUES ('DELETE_ATTEMPT', OLD.emp_id, OLD.salary);
END$$
DELIMITER ;

-- Example to try
DELETE FROM employees WHERE emp_id = 2;
SELECT * FROM audit_log;
\end{lstlisting}

\subsubsection{}
\begin{center}
    \includegraphics[width=0.8\textwidth]{10.png}
\end{center}

% 11
\section{After\_Delete}
\textbf{Task:} Create a trigger that copies deleted employee records into deleted\_employees.

\subsection{}
\begin{lstlisting}
DELIMITER $$
CREATE TRIGGER after_emp_delete_copy
AFTER DELETE ON employees
FOR EACH ROW
BEGIN
 INSERT INTO deleted_employees(emp_id, emp_name)
 VALUES (OLD.emp_id, OLD.emp_name);
END$$
DELIMITER ;

-- Example to try
DELETE FROM employees WHERE emp_id = 3;
SELECT * FROM deleted_employees;
\end{lstlisting}

\subsubsection{}
\begin{center}
    \includegraphics[width=0.8\textwidth]{11.png}
\end{center}

% 12
\section{After\_Delete}
\textbf{Task:} Create a trigger that logs every deletion in the audit\_log table.

\subsection{}
\begin{lstlisting}
DELIMITER $$
CREATE TRIGGER after_emp_delete_log
AFTER DELETE ON employees
FOR EACH ROW
BEGIN
 INSERT INTO audit_log(action, emp_id, old_salary)
 VALUES ('DELETE', OLD.emp_id, OLD.salary);
END$$
DELIMITER ;

-- Example to try
DELETE FROM employees WHERE emp_id = 4;
SELECT * FROM audit_log;
\end{lstlisting}

\subsubsection{}
\begin{center}
    \includegraphics[width=0.8\textwidth]{12.png}
\end{center}

\end{document}
